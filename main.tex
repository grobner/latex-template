\documentclass[dvipdfmx,10.5pt,a4paper]{jsarticle}
%
\usepackage{amsmath,amssymb}
\usepackage{bm}
\usepackage[dvipdfmx]{graphicx}
\usepackage{ascmac}
\usepackage{mathtools}
\usepackage{comment}
\usepackage{here}
\usepackage{url}
\usepackage[separate-uncertainty]{siunitx}
\usepackage[subrefformat=parens]{subcaption}
\usepackage{pdfpages}
\usepackage{physics}
\usepackage{algorithm, algpseudocode}
% \usepackage{algpseudocode}
%\usepackage{jtygm} %参考文献リストでの斜体を表示するのにたぶん必要
\renewcommand{\bibname}{参考文献} %関連図書→参考文献に変える

%\usepackage{color}
\setlength{\textwidth}{\fullwidth}
\setlength{\textheight}{40\baselineskip}
\addtolength{\textheight}{\topskip}
\setlength{\voffset}{-0.2in}
\setlength{\topmargin}{0pt}
\setlength{\headheight}{0pt}
\setlength{\headsep}{0pt}
%\pagestyle{empty}
%


\newtheorem{thm}{定理}
\newtheorem{dfn}{定義}
\renewcommand{\algorithmicrequire}{\textbf{Input:}}
\renewcommand{\algorithmicensure}{\textbf{Output:}}

%

\begin{document}
\tableofcontents
\clearpage


\begin{algorithm}[t]
    \caption{なんとかなアルゴリズム}
    \label{alg:hoge}
    \begin{algorithmic}
    \footnotesize{
      \Require{$x$:なんとか,$\tau$:閾値}
      \Ensure{$x$:ほにゃらら}
      \Function{hoge}{$x$}
        \State return $x$
      \EndFunction
    }
    \end{algorithmic}
  \end{algorithm}
    
うんち\cite{book_label}

\bibliographystyle{junsrt}
\bibliography{sankou}

\clearpage

%\appendix
%\input{huroku.tex}

\end{document}
